\chapter{結論}
\label{chapter:conclusion}
本研究では, エンドノード間通信において複数のインタフェースを用いて複数の経路がある今日のデータセンターネットワークでのネットワーク内通信における,
ショートフローの遅延問題について, MPTCPを用いた経路切替手法によってFCTの短縮化を行う手法を提案した. 
遅延の主な原因の一つであるキューイング遅延が発生しているスイッチに対して直接変更を加える従来手法では, 実環境への適用に大きな課題があった. 
そこで, 提案手法では, エンドノードのOSに対してのみ変更を加えることで, 通信性能の改善を目指した. 
併せて, 実際のデータセンターが生成するトラフィックの一例として,
データセンターで広く利用されている並列分散処理アプリケーションのトラフィックを観測することで,
サイズの大きいスループット指向なロングフローとサイズの小さいレイテンシ指向なショートフローが一つのリンクに混在して通信を行っていることを明らかにし,
それによる通信性能障害が生じる技術的背景を考察した. 
そして, 指向毎に異なる通信経路を利用するレーンモデルを提案した. 
また, OSレベルから一つの通信を見た際に, そのフローの指向をフロー持続時間を用いて分類し, 経路の状況に合わせた経路切替手法を提案した. 

提案手法をOSに実装し, シミュレーションを用いた評価を行い, ロングフローとショートフローが混在する環境下で,
それぞれの指向毎に経路を切り替える制御を行うことで, ショートフロー通信を短縮化することを示した. 

今後の課題としては, 提案手法の適用範囲の拡大とエンドノードに対する通信性能の向上が考えられる. 
適用範囲の拡大としては, Incast問題への対応が考えられる. 
並列分散処理を運用するデータセンターにおいて, Queue buildupと並ぶ大きな課題であるIncast問題に対し, 本提案手法では改善することはできない. 
これは指針として, エンドノードのみに対するアプローチを行う以上, 予め経路の状況を把握しておき, それに応じて最適な経路制御を行うことが困難であるためである. 
また, 現状MPTCP実装がサイズの小さい通信には複数経路を利用せず, 初めにコネクションを確立した経路のみで通信を完了するためである. 
この課題の解決として, エンドノードへのアプローチとしてOS側で通信履歴を保持しておき, 経路を切り替える手法,
スイッチに対するアプローチとしてSDNを用いて全ルーティングを管理する手法が考えられる. 
前者については, 通信の発生頻度に依存する経路決定の正確さ, 後者では, スケーラビリティの問題があり, それらを解決するためのフロー制御を考えていきたい. 
また近い将来の通信帯域の広帯域化によるエンドノードのボトルネック問題があり, CPU負荷分散手法として, MPTCPによる経路切替による,
インタフェース切替手法が期待されている.
このように, 今後ネットワークのみならずエンドノードの性能まで拡大した改善手法が求められていると考えられ,
上記の課題を解決することでデータセンター性能の改善が実現できると考えられる. 
